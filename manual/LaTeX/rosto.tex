
\thispagestyle{empty}
\begin{center}
	UNIVERSIDADE ESTADUAL DE CAMPINAS\\
	FACULDADE DE ENGENHARIA EL'ETRICA\\
	DEPARTAMENTO DE COMUNICA,C~OES

	\ruley{50mm}

	{\Huge
		METODOLOGIAS DE AVALIA,C~AO 

		DE ALGORITMOS DE 

		CODIFICA,C~AO DE VOZ

 		\rulex{1mm}
	}

	SIM~AO FERRAZ DE CAMPOS NETO

	Orientador: Prof.Dr. F'ABIO VIOLARO

\end{center}

\ruley{10mm}

\rulex{1mm} \hfill  
	      \parbox[t]{80mm}{Banca Examinadora:\\
		{\em F'abio Violaro} ({\normalsize\sf UNICAMP})\\ 
		{\em Abraham Alcaim} ({\normalsize\sf CETUC})\\ 
		{\em Jo~ao Marcos Travassos Romano} ({\normalsize\sf UNICAMP})\\
		{\em Leonardo Mendes} ({\normalsize\sf UNICAMP})
	      }

\ruley{5mm}

\rulex{1mm} \hfill \parbox[t]{80mm}{Tese apresentada `a Faculdade de 
					 Engenharia El'etrica da
					 Universidade Estadual de
					 Campinas como parte dos
					 requisitos para a obten,c~ao
					 do t'itulo de MESTRE EM
					 ENGENHARIA EL'ETRICA.
					}

\vfill

\begin{center}
			Campinas, Abril de 1993
\end{center}

%============================================================================
\dummypage
%============================================================================
\newpage
\thispagestyle{empty}

\begin{center}

	\ruley{50mm}

	{\Huge
		METODOLOGIAS DE AVALIA,C~AO 

		DE ALGORITMOS DE 

		CODIFICA,C~AO DE VOZ

 		\rulex{1em}
	}

	SIM~AO FERRAZ DE CAMPOS NETO

	Orientador: Prof.Dr. F'ABIO VIOLARO

\end{center}

\vfill

\begin{center}
			Campinas, Abril de 1993
\end{center}

%============================================================================
\newpage
\thispagestyle{empty}

\ruley{30mm}

\begin{center}
  \framebox[125mm][c]{
	\parbox{105mm}{ 
		{\bf Campos Neto, Sim~ao Ferraz de} \ruley{2em}\\
		\rulex{5mm} Metodologias de Avalia,c~ao de Algoritmos de
		Codifica,c~ao de Voz / Sim~ao Ferraz de Campos Neto, 1993.

		\rulex{5mm} 159 p'aginas.

		\rulex{5mm} Tese (Mestrado) - Universidade Estadual 
		de Campinas, 1993.

		\rulex{5mm} 1. Processamento e An'alise de Voz \ \ 2.codecs \ \ 
		3. An'alise Estat'istica. I. T'itulo

		\rulex{1mm} \hfill 621.38043 \smallskip
	}
  }
\end{center}

\ruley{40mm}

Copyrigh \copyright 1992 Sim~ao Ferraz de Campos Neto. 

Este documento foi editorado com o sistema \LaTeX\ e impresso numa impressora laser LPS20 (DEC). Vers~ao de \hoje. Publicado pela Unicamp. C'opias desta tese podem ser obtidas:

{\centering \begin{minipage}{100mm}
                   Biblioteca da Faculdade de Engenharia El'etrica\\
                   {\em (Pr'edio da Biblioteca Central)}\\
                   Universidade Estadual de Campinas -- Unicamp\\
                   13100-000 Campinas SP
            \end{minipage}\\*[10mm]
}
 
%============================================================================
\newpage
\thispagestyle{empty}

\ruley{60mm}
\begin{center}
		\bf Resumo
\end{center}


\begin{quote}
\em
	Neste trabalho s~ao apresentados diversos aspectos relacionados
	`a avalia,c~ao da qualidade subjetiva e objetiva de algoritmos
	de codifica,c~ao de voz, como metodologia de testes,
	infra-estrutura, descri,c~ao de algoritmos de refer^encia e de
	medidas objetivas. Este trabalho 'e importante por fornecer
	subs'idios para a implementa,c~ao de metodologias efetivas que
	garantam a qualidade de codificadores do sinal de voz quando
	utilizados na rede telef^onica. Ap'os a parte tutorial deste
	trabalho, analisam-se os resultados de um dos testes subjetivos
	para a l'ingua portuguesa realizados durante a padroniza,c~ao
	da hoje Recomenda,c~ao CCITT G.728 e os resultados de medidas
	objetivas de qualidade, bem como a sua capacidade de estimar a
	qualidade subjetiva.

\end{quote}

%============================================================================
\newpage
\thispagestyle{empty}

\ruley{60mm}
\begin{center}
		\bf Abstract
\end{center}


\begin{quote}
\em
	Many topics related to the subjective and objective assessment
	of speech quality for speech coding algorithms are presented in
	this work, such as: test methodologies, laboratorial
	facilities, description of reference algorithms, and objective
	measures. This work is important because it gives the basement
	for the development of effective methodologies that assure the
	quality of speech coders for use in the telephone network.
	After the tutorial part, it is presented the results of one of
	the subjective tests for the Portuguese language, made during
	the standardization of the present CCITT Recommendation G.728,
	and the results of some objective measures of quality, as well
	as their capacity of estimating the subjective quality.

\end{quote}

%============================================================================
\newpage
\thispagestyle{empty}

\ruley{200mm}

\begin{flushright}
	\em `A minha filha 'Iris e \\ `a minha esposa Yoshiko.
\end{flushright}

%============================================================================
\newpage
\thispagestyle{empty}

\ruley{50mm}
\begin{center}
		\bf Agradecimentos
\end{center}


\begin{quote}
\em
	Este trabalho n~ao poderia ter sido realizado sem a ocorr^encia
	de diversos fatos. Primeiro, a decis~ao estrat'egica do
	CPqD/Telebr'as, em 1989, de participar dos testes subjetivos
	para o codificador CCITT a 16 kbit/s como um mecanismo de
	forma,c~ao de recursos humanos relacionados `a avalia,c~ao de
	qualidade de algoritmos de codifica,c~ao de voz. Fui escolhido
	para acompanhar esta atividade e devo deixar meu agradecimento
	ao Eng.Jos'e Sindi Yamamoto pela sua confian,ca e pelas
	discuss~oes sempre produtivas. Tamb'em n~ao houvesse a
	disposi,c~ao em enveredar por novos caminhos, n~ao teria
	contado com a orienta,c~ao fiel do Prof.Dr.F'abio Violaro, que
	dedicadamente permitiu evoluir este trabalho, sempre
	incentivando minha liberdade criativa (mas sempre me pedindo
	para n~ao deixar o trabalho crescer muito...).  Tamb'em contei
	com a colabora,c~ao dos colegas, que gentilmente cederam suas
	vozes para a gera,c~ao do material de voz para os testes
	subjetivos, ou seus ouvidos e julgamento para as sess~oes de
	testes subjetivos. Para esta 'ultima atividade, contei tamb'em
	com o apoio de amigos e parentes, que muitas vezes vieram ao
	Centro em feriados e finais de semana. A estes, meus sinceros
	agradecimentos. Aos colegas de trabalho, preciso agradecer a
	compreens~ao e desculpar-me pelos maus humores, sempre
	freq"uentes ao longo desta jornada. E, em casa, agradecer o
	carinho constante de minha esposa e filha, sem cujo apoio este
	trabalho por certo malograria.
\end{quote}

%============================================================================
%%%%%\dummypage

%============================================================================
\newpage
\thispagestyle{plain}
\pagenumbering{roman}
\setcounter{page}{1}

\begin{center}
	\LARGE\bf Preface
\end{center}

\ruley{10mm}

In the late 1980's, various methods, available on a regional level,
were proposed for the coding of speech at 16 kbit/s. Not only was the
proliferation of regional and incompatible algorithms detrimental to
global interworking, but the modest quality of these algorithms
threatened to impose undue constraints on overall network transmission
planning. In this context, a question arose within the International
Telephone and Telegraph Consultative Committee (CCITT) as to whether it
was possible to define a new ``universal'' 16 kbit/s algorithm that
could be used worldwide for all potential applications and still
achieve network transparency. This question led to a set of highly
challenging requirements which spurred speech researchers to
investigate innovative ways to meet the challenge.

From the beginning, this CCITT effort required close coordination
between experts in speech coding, and experts on transmission
performance who produced a set of requirements and methodologies for
subjective and objective testing of voice codecs. The subjective test
plan called for extensive multilingual tests and required the
collaboration of laboratories in many countries, while the objective
test plan called for non-voice signaling performance assessments as
well as objective measurements using voice signals.

In May 1922, following a four-year effort, CCITT Recommendation G.728
on Low-Delay Code-Excited Linear Prediction (LD-CELP) was finally
approved for the ``telephone-quality'' coding of speech at 16 kbit/s.
It was through the definition and implementation of extensive and
detailed test methodologies, both subjective and objective, that the
determination of the LD-CELP algorithm's ability to meet the specified
performance requirements was possible.

It is noted that this was a pioneering experiment whereby
standardization stimulated and led, rather than followed or
consolidated, innovative developments in speech coding and testing. It
is also noted that the test methodologies developed in this process are
expected to be a model for years to come, as evidenced by their planned
use in CCITT's 8 kbit/s speech coding standardization effort.

It is with this preface in mind that is my pleasure to introduce this
present thesis, which I consider to provide a significant insight into
the selection of speech coding technologies for the ``network of the
future''.

\begin{flushright}
Spiros Dimolitsas.\\
Chairman of the CCITT Ad Hoc Group \\
of Experts on 16 kbit/s Speech Coding.\\
Comsat Laboratories, April of 1993.
\end{flushright}

%-----------------------------------------------------------------------
\newpage
\thispagestyle{empty}

\mbox{\ruley{10cm}}


