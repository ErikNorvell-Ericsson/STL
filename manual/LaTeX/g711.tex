
%=============================================================================
% ...... THIS IS chapter{G.711: The ITU-T 64 kbit/s log-PCM algorithm } ......
% Revisions: 
% 04.Jan.1996 - Peter Kroon revision incorporated
% 22.Feb.2001 - Edits in example section
%
%=============================================================================
\chapter{G.711: The ITU-T 64 kbit/s log-PCM algorithm}
%=============================================================================

In the early 1960's an interest was expressed in encoding the analog signals
in telephone networks, mainly to reduce costs in switching and
multiplexing equipments and to allow the integration of communication
and computing, increasing the efficiency in operation and maintenance
\cite{Qual-meas-tel-sys}.

In 1972, the then CCITT published the Recommendation \textcolor{blue}{ITU-T} G.711 that
constitutes the principal reference as far as transmission systems
are concerned \cite{G.711}. The basic principle of the algorithm is
to code speech using 8 bits per sample, the input voiceband signal
being sampled at 8 kHz, keeping the telephony bandwidth of
300--3400 Hz. With this combination, each voice channel requires
64 kbit/s.

\section{Description of the algorithm}

The idea behind the digitalization of the network involved a
compromise: use as far as possible the existing infrastructure; this imposes a
bandwidth limitation for the bit-streams of coded signals. A rate of 64
kbit/s was found to be reasonable.

If one thinks of using the most natural quantization scheme, one will
choose linear quantization. But one drawback of this approach is that
the signal-to-noise ratio (SNR) varies with the amplitude of the input
signals: the smaller the amplitude, the smaller the SNR. And, from the
quality point of view, if a signal has a wide variance, or a variance
that changes with time (as in the case of speech signals), the SNR
will also change, resulting in a wide-varying quality of the
system.

To avoid this problem, one can use logarithmic quantization, which
will result into a more uniform quantization noise. With this in
mind, several studies were carried out in late 1960's to choose a
good algorithm for this purpose. This led to the definition of two
transmission schemes, one using the $\mu$ law compression
characteristic: 
\[ 
   c(x) = x_{max} \frac{\ln(1 + \mu |x|/x_{max}) }{
                        \ln(1+\mu) } sgn(x) 
\] 
and the other using the A law compression characteristic: 
\[ 
   c(x) = \left \{ \begin{array}{ll}
          {\displaystyle A|x| \over \displaystyle 1 + \ln(A) } sgn(x), 
                &\mbox{for } 0 \leq \Frac{|x|}{x_{max}} \leq \Frac{1}{A}\\
          x_{max} {\displaystyle 1 + \ln(A|x|/x_{max}) \over 
                   \displaystyle 1 + \ln(A) } sgn(x),
                &\mbox{for \ruley{2em}} 
                   \Frac{1}{A} \leq \Frac{|x|}{x_{max}} \leq 1
          \end{array}
       \right .
\]

Both characteristics behave as linear for small amplitude signals
(being then equivalent to a linear quantization scheme), but are truly
logarithmic for large signals. In fact, for large signals the SNR is:
\[
    \mbox{\em SNR}_\mu = 6.02 B + 4.77 - 20 \log_{10} (\ln(1+\mu))
\]
and
\[
    \mbox{\em SNR}_A = 6.02 B + 4.77 - 20 \log_{10} (1 + \ln A)
\]
where $B$ is the number of bits used for quantization.

The ITU chose the values $A=87.56$ and $\mu=255$ for the G.711
standard, together with 8 bits per sample, what leads the latter two
equations to:
\[
    \mbox{\em SNR}_\mu = 6.02 B - 9.99 = 38.17 dB
\]
and
\[
    \mbox{\em SNR}_A = 6.02 B - 10.1 = 38.06 dB
\]

The G.711 standard does not specify the law as defined above, but
rather uses a good linear-piecewise approximation for 8 bit samples,
which has easier implementation (in hardware), as well as other
properties (see \cite[p.229]{Jayant-Noll}).

This approximation uses bit 1 for sign (1 for positive, 0 for
negative), bits 2--4 to indicate a segment, and bits 5--8 for
level\footnote{\SF Please note that the bit numbering in the G.711 is
the reversal of the commonly used in computer languages, G.711's bit 1
corresponding to common-sense's (most significant) bit 7, and G.711's
bit 8 to the normal least significant bit 0, respectively.}. Within
each segment, the quantization is linear (4 bits, or 16 levels), having
15 segments of distinct slopes for $\mu$ law, and 13 for A law.

The A law works with signals in the range from -4096 to 4096,
implying in a range of 13 bits. As for the $\mu$ law, the linear
signals are accepted in the range -8159 to 8159, which is represented
by 14 bits. Besides this, in the dynamic range sense, A and $\mu$ law
are equivalent to 12 and 13 bit linear quantization, respectively.

One detail for the A law is that the even bits are inverted. The reason
for this comes from problems observed (before the standardization of
the line code HDB3) in transmission systems when long sequences of
zeros happen, because small amplitudes, in A law, to be coded mostly
using `0' bits. With this bit-inversion, long sequences of bits `0'
becomes less probable, thus improving performance.

The conversion rule for A/$\mu$ law from/to linear is described in
terms of tables in G.711. A good reason for this is that there is no
closed form for the compression of linear samples (although it is
possible to find a closed formulae for the expansion algorithm). 
Hence, two implementations are possible: table look-up, and algorithmic.
For in-chip (LSI) implementations, the first one may be preferred,
because it is simpler to implement, at the cost of a wider chip area.
For other applications, such as using Digital Signal Processors (DSPs),
or software implementations, table look-up would occupy too much
memory, and the algorithmic solution would be preferred.

\section{Implementation}

This implementation of the G.711 can be found in the module {\tt
g711.c}, with prototypes in {\tt g711.h}.

For the reason explained before, an algorithmic approach to the G.711
was followed. For the compression routines, first the samples are
converted from two's complement to signed magnitude
notation\footnote{\SF Using the samples as two's complement in the
compression algorithm is a very common error whose effects are only
noticeable for small amplitude signals. Our approach agrees to the one
in G.726\cite{G.726}, block {\em compress}.}. So, a segment
classification is done, and then the linear quantization of a certain
number of bits of the input sample, that depends on the segment number
(e.g., for A law, segment 1 uses a factor of 2:1, 2 a 4:1, etc.) is
carried out. Finally, the sign of the sample is added. The expansion
routines are even simpler: find the sign, get the mantissa and the
exponent, and compute the linear sample.

One important point here is that, following UGST Guidelines, linear
input samples must be left-justified {\tt short}s. With this approach,
the knowledge of the 0 dB reference for the file is simplified, and the
need of having to apply different normalization factors to files if
they are to be coded by A or $\mu$ law is eliminated\footnote{\SF In
the case of stand-alone tools, this would mean that two copies of the
same file should be available!}. As an example, suppose that we want to
process a speech file $\cal X$ by the G.711 at an input level of -20
dBov for both A and $\mu$ law. Then, if the sample representation is
right-justified, and a factor $f$ brings a file's level to -20 dBov for
$\mu$ law, then for A law the factor will be $2.f$, due to the
difference in input signal's dynamic range of both laws (4096 and 8159,
respectively). On the other hand, if the samples are left-justified,
the factor is only one, and the routines will only look at the 13 or 14
most significant bits of the 16-bit word, for A and $\mu$ law,
respectively. In other words, the peak value for linear and A/$\mu$ law
is the same, therefore one factor is sufficient.

Compliance tests to this code have been done using a ramp file having
the full excursion of the dynamic range for each of the laws, and
examining the compressed and expanded samples against the values
expected in tables 1a, 1b, 2a, and 2b of Recommendation \textcolor{blue}{ITU-T} G.711 (see
\cite{G.711}). Another test done exploits the synchronous property of
the G.711 scheme. Only samples from column 7 of G.711 tables 1 and 2
were used. These values are transparent to quantization. Hence, if
the coding was done properly, output samples should match exactly the
original ones.

The compression functions are {\tt alaw\_compress} and {\tt
ulaw\_compress}, and the expansion functions are {\tt alaw\_expand} and
{\tt ulaw\_expand}. In the next part you find a summary of calls to
these functions.

\subsection{{\tt alaw\_compress} and {\tt ulaw\_compress}}

{\bf Syntax: } 

{\tt
\#include "g711.h"\\
void alaw\_compress
         (\ttpbox{110mm}{
            long {\em smpno}, short {\em *lin\_buf}, short {\em *log\_buf})
         }\\
void ulaw\_compress
         (\ttpbox{110mm}{
            long {\em smpno}, short {\em *lin\_buf}, short {\em *log\_buf})
         }
}

{\bf Prototype: }       g711.h

{\bf Description: }

{\tt alaw\_compress} performs A law encoding rule according to
Recommendation \textcolor{blue}{ITU-T} G.711, and {\tt ulaw\_compress} does the same for $\mu$
law. Note that input samples shall be left-justified, and that the
output samples are right-justified with 8 bits.

{\bf Variables: }
\begin{Descr}{\DescrLen}
\item[\pbox{20mm}{\em smpno}] %\rulex{1mm}\\
               Is the number of samples in lin\_buf.

\item[\pbox{20mm}{\em lin\_buf}] %\rulex{1mm}\\
               Is the input samples' buffer; each {\tt short} sample 
               shall contain linear PCM (2's complement, 16-bit wide) 
               samples, left-justified.

\item[\pbox{20mm}{\em log\_buf}] %\rulex{1mm}\\
               Is the output samples' buffer; each {\tt short} sample 
               will contain right-justified 8-bit wide valid A or $\mu$ 
               law samples. 
 
\end{Descr}
        
        {\bf Return value: }        None.


\subsection{{\tt alaw\_expand} and {\tt ulaw\_expand}}

{\bf Syntax: } 

{\tt
\#include "g711.h"\\
void alaw\_expand 
         (\ttpbox{110mm}{
            long {\em smpno}, short {\em *log\_buf}, short {\em *lin\_buf})
         }\\
void alaw\_expand 
         (\ttpbox{110mm}{
            long {\em smpno}, short {\em *log\_buf}, short {\em *lin\_buf})
         }
}

{\bf Prototype: }       g711.h

{\bf Description: }

{\tt alaw\_expand} performs A law decoding rule according to
Recommendation \textcolor{blue}{ITU-T} G.711, and {\tt ulaw\_expand} does the same for $\mu$
law. Note that output samples will be left-justified, and that the
input samples shall be right-justified with 8 bits.

{\bf Variables: }
\begin{Descr}{\DescrLen}
\item[\pbox{20mm}{\em smpno}] %\rulex{1mm}\\
               Is the number of samples in log\_buf.

\item[\pbox{20mm}{\em log\_buf}] %\rulex{1mm}\\
               Is the input samples' buffer; each {\tt short} sample 
               shall contain right-justified 8-bit wide valid A or $\mu$ 
               law samples.

\item[\pbox{20mm}{\em lin\_buf}] %\rulex{1mm}\\
               Is the output samples' buffer; each {\tt short} sample 
               will contain linear PCM (2's complement, 16-bit wide) 
               samples, left-justified.

\end{Descr}
        
        
        {\bf Return value: }        None.


\section{Tests and portability}  

Portability may be checked by running the same speech file in a
proven platform and in a test platform. Files processed this way
should match exactly. Source and processed reference files for
portability tests are provided in the STL distribution.

These routines had portability tested for VAX/VMS with VAX-C and gcc, 
MS-DOS with Turbo C v2.0, HPUX with gcc, and Sun-OS with Sun-C.


%-----------------------------------------------------------------
\section{Example code}

%.................................................................
\subsection {Description of the demonstration program}

One program is provided as demonstration program for the G.711 module,
g711demo.c.

Program {\tt g711demo.c} accepts input files in 16-bit linear PCM
format for compression operation and produces files in the same format
after the expansion operation. The compressed signal will be in
16-bit, right adjusted format, according to the logarithmic law
specified by the user. Three operations are possible: linear in,
linear out ({\em lili}) linear in, logarithmic out ({\em lilo}), or
logarithmic in, linear out ({\em loli}).

%.................................................................
\subsection {Simple example}

The following C code gives an example of companding using either the
A- or $\mu$-law functions available in the STL. 

{\tt\small 
\begin{verbatim}
#include <stdio.h> 
#include "ugstdemo.h" 
#include "g711.h"

#define BLK_LEN 256
#define QUIT(m,code) {fprintf(stderr,m); exit((int)code);}

main(argc, argv)
  int             argc;
  char           *argv[];
{
  char            law[4];

  char            FileIn[180], FileOut[180];
  short           tmp_buf[BLK_LEN], inp_buf[BLK_LEN], out_buf[BLK_LEN];
  FILE           *Fi, *Fo;
  void          (*compress)(), (*expand)(); /* pointer to a function */

  /* Get parameters for processing */
  GET_PAR_S(1, "_Law (A,u): ................... ", law);
  GET_PAR_S(2, "_Input File: .................. ", FileIn);
  GET_PAR_S(3, "_Output File: ................. ", FileOut);

  /* Opening input and output LOG-PCM files */
  Fi = fopen(FileIn, RB);
  Fo = fopen(FileOut, WB);

  /* Choose compression/expansion routinies according to the law */
  if (toupper(law[0])=='A')
  {
     compress = alaw_compress;
     expand = alaw_expand;
  }
  else if (tolower(law[0])=='u')
  {
     compress = ulaw_compress;
     expand = ulaw_expand;
  }
  else 
    QUIT("Bad law chosen!\n",1);

 /* File processing */
  while (fread(inp_buf, BLK_LEN, sizeof(short), Fi) == BLK_LEN)
  {
    /* Process input linear PCM samples in blocks of length BLK_LEN */
    compress(BLK_LEN, inp_buf, tmp_buf);

    /* Process log-PCM samples in blocks of length BLK_LEN */
    expand(BLK_LEN, tmp_buf, out_buf);

    /* Write PCM output word */
    fwrite(out_buf, BLK_LEN, BLK_LEN, sizeof(short), Fo);
  }

  /* Close input and output files */
  fclose(Fi);
  fclose(Fo);
  return 0;
}
\end{verbatim}
}
