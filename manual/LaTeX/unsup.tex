
\chapter{Unsupported tools}

This Appendix to the ITU-T Software Tool Library (STL) Manual
describes the unsupported tools provided in the ITU-T
STL. \textcolor{blue}{The tools are named as ``unsupported'' because
they do not follow the initial modularity concept of STL.} These
tools are provided ``as is'' and without any warranties or implied
suitability to use. However, any feedback on problems with these tools
will be welcome and accomodated as possible, as will any improvements
made which can be shared and incorporated in the STL.


\section{Source code}
%~~~~~~~~~~~~~~~~~~~~~

\begin{Descr}{40mm}
\item[asc2bin.c:]
        converts decimal or hex ASCII data into short/long/float
                or double binary numbers. Input data must be one
                number per line.

\item[astrip.c:]
        strips off a segment of a file. Can operate on block or
                sample-based parameters and can apply windowing to the
                borders of the extracted segment. Tested in Unix/MSDOS.

\item[bin2asc.c:]
        converts short/long/float or double binary numbers into
                octal, decimal or hex ASCII numbers, printing one per
                line. For Unix/MSDOS.

\item[compfile.c:]
        compare word-wise binary files. For VMS/Unix/MSDOS.

\item[dumpfile.c:]
        dump a binary file. For VMS/Unix/MSDOS.

\item[chr2sh.c:]
           convert {\tt char}-oriented files to \short-oriented (16-bit
           words) files by padding the upper byte of each word of
           the output file with zeros. For Unix/MSDOS.

\item[endian.c:]
  verify whether the current platform is big or little endian
  (i.e. high-byte first or low-byte first). For Unix/MSDOS.

\item[fdelay.c:]
        flexibly introduce delay into a file. Delay
        can be specified in value and length, or can be taken
        from a user-specified file. For Unix/MSDOS.

\textcolor{blue}{%
\item[filcom\_w16.c:]
  compares 16-bit integer files, giving on the following information:
  file sizes, the location of the first difference, the largest
  difference, histogram of the first N difference value, global and
  segmental SNR, and block location of the lowest/highest SNR. For
  Unix/MSDOS.
%
}

\item[g728-vt:]
            a directory with software tools for use with the G.728
            floating-point verification package. Not all tools are
            functional; preserved here for future reference.

\item[getcrc32.c:]
         32-bit CRC calculation function and program (depending on
         how it is compiled). Uses the same polynomial as
         ZIP. Checked for portability across a number of
         platforms. Makefile compiles it into an executable called
         crc. For Unix/MSDOS.

\item[measure.c:]
        measure statistics/CRC for a bunch of files.  For
        VMS/Unix/MSDOS.

\item[oper.c:]
            implement arithmetic operation on two files: add,
            subtract, multiply or divide two files applying
            scaling factors (linear or dB), and adding a DC
            level. For Unix/MSDOS.

\item[pshar:]
        a directory with makefiles, readme, source code and
        test files for a portable shell archiving/dearchiving
        program compatible with Unix the shar utility. Very
        simple and useful, in especial for MSDOS and VMS
        systems. See the directory for more details.

\item[sb.c:]
        swap bytes for word-oriented files. For
        VMS/Unix/MSDOS.

\item[sh2chr.c:]
           convert \short-oriented (16-bit words) files to {\tt
           char}-oriented files by ignoring the upper byte of each
           word of the input file. For Unix/MSDOS.

\item[sine.c:]
        generate a sinewave file for a given speco of
        AC/DC/phase/ frequency/sampling frequency values. For
        VMS/Unix/MSDOS.

\item[sub-add.c:]
        subtract/add files (depending on the compilation, see
        makefiles). For VMS/Unix/MSDOS.

%\item[\textcolor{blue}{stereoop.c}] \textcolor{blue}{Basic
%        stereo operations on 16-bit audio files. The functionalities
%        include: }

\item[xencode.c:]
        uuencode compatible with auto-break/sequencing for
        long files and CRC calculation for error
        detection. Not functional under MSDOS 6.22.

\item[xdecode.c:]
        uudecode compatible with auto-break/sequencing for
        long files and CRC calculation for error
        detection. Not functional under MSDOS 6.22.

\item[g729e\_convert\_synch.c:]
        convert G.192 word oriented analysis 
        frames without SYNCH\_WORD into a G.192 word oriented 
        file with SYNCH\_WORD and zeroed payload bits. 
         
\end{Descr}


\section{Scripts}
%~~~~~~~~~~~~~~~~

\begin{Descr}{40mm}
\item[rm.bat]
                ``fake'' deletion utility that tries to emulate the
        basic functionality of the Unix command {\tt rm}, that
        deletes multiple files specified in the command
        line. Should be put in the path, unless
        a version of {\tt rm} is already available.
\item[swapover.bat]
                MSDOS batch script for byte-swapping multiple
                files. Uses sb.c.
\item[swapover.sh]
                Unix script for byte-swapping multiple files. Uses sb.c.
\end{Descr}

\section{Makefiles}
%~~~~~~~~~~~~~~~~~~

\begin{Descr}{40mm}
\item[makefile.tcc] for Borland [bt]cc C/C++ compiler
\item[makefile.djc] for MSDOS djc port of gcc
\item[makefile.unx] for Unix make
\item[makefile.cl]      for MS Visual C command-line compiler
\end{Descr}


\section{Test files}
%~~~~~~~~~~~~~~~~~~~

\begin{Descr}{40mm}
\item[tstunsup.zip]
                zip archive with test files for testing some of the
                unsupported tools: \\
        {\tt\begin{verbatim}
   cf:
     3200    cftest1.dat
     3200    cftest2.dat
     3200    cftest3.dat
      186    delay-15.ref
      186    delay-a.ref
      214    delay-u.ref
      186    delaydft.ref
      200    delayfil.ref

   sb:
      100    bigend.src
      100    litend.src

   xencode and xdecode:
     9182    voice.ori
     8705    voice.uue
     2093    printme.eps
     3795    printme.uue
     5368    voice01.uue
        \end{verbatim}}
                It is necessary to have unzip/pkunzip/Winzip installed
                for extraction.
\end{Descr}
