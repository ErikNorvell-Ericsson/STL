%=============================================================================
% ..... THIS IS chapter{TRUNCATE: ITU-T bitstream truncation tool } .....
%     Apr.2005
% ... Authors :
%        Cyril Guillaum� & St�phane Ragot - stephane.ragot@francetelecom.com
% Apr.2005 - Created
% Oct.2005 - Editorial changes (Simao Campos)
%=============================================================================
\chapter{G.711 Appendix I: A high quality low-complexity 
algorithm for packet loss concealment with G.711.}
%=============================================================================

%----------------------------------------------------------------------
\section{Introduction}
%----------------------------------------------------------------------

Packet Loss Concealment (PLC) algorithms hide transmission losses in
audio systems where the input signal is encoded and packetized at a
transmitter, sent over a network, and received at a receiver that
decodes the packet and plays out the output. G.711 Appendix I
\cite{G711-appendix-I}, approved by ITU-T in September 1999, describes
a high quality, low complexity PLC algorithm designed for use with
G.711.

%----------------------------------------------------------------------
\section{Description of the algorithm}
%----------------------------------------------------------------------

A brief description of the PLC algorithm is given. A more extensive
presentation can be found in Section I.2, ``Algorithm description'',
of G.711 Appendix I \cite{G711-appendix-I}.

The PLC algorithm is inserted after the G.711 decoder at the
receiver. The algorithm is designed to work with 10 ms frames, or 80
samples per frame at 8 KHz sampling. An external mechanism is needed
to signal when packets are lost. Since speech signals are often
locally stationary, the signals recent history is used to generate a
reasonable approximation to lost frames. If the losses are not too
long, and do not land in a region where the signal is rapidly
changing, the losses may be inaudible after concealment.

When a frame is received the decoded speech is given to the PLC
algorithm. Received frames are saved in a 48.75 ms circular history
buffer, and the output is delayed by 3.75 ms (30 samples).

When a packet is lost the concealment algorithm starts synthetic
signal generation. First the pitch is estimated by finding the peak
of the normalized autocorrelation of the most recent 20 ms of speech
in the history buffer with the previous speech at taps from 5 to 15
ms. Using the pitch estimate, the most recent pitch period from the
history buffer is repeated for the duration of the first lost frame
(10 ms). If the pitch estimate is longer than 10 ms, only a portion
of the most recent pitch period will be used in the first lost
frame. A 1/4 pitch period overlap add (OLA) with a triangular window
is performed at all repetition boundaries, including the transition
between the last received frame and the start of the synthetic
signal.

If consecutive frames are lost, the number of pitch periods used to
generate the synthetic signal is increased by one pitch period at
the start of the 2nd and 3rd lost frames. When the number of pitch
periods is increased, the output is smoothly transitioned to the
oldest used pitch period of the history signal with an additional
1/4 pitch period OLA. Increasing the number of pitch periods reduces
the number of unnatural harmonic artifacts in the concealed speech
for long losses. The algorithm does not distinguish between voiced
and un-voiced speech and uses the same procedure for both types of
speech.

At the start of the first received frame after a loss, the synthetic
signal generation is continued and OLAed with the received speech.
This OLA window length increases with the length of the loss. For
single frame losses it is 1/4 of the estimated pitch period. 4 ms
are added for each additional consecutive lost frame, up to a
maximum of 10 ms.

If the loss exceeds 10 ms the synthetic signal is also linearly
attenuated at the rate of 20\% per frame. If the loss exceeds 60 ms
the synthesized signal is set to silence.


%----------------------------------------------------------------------
\section{Implementation}
%----------------------------------------------------------------------

%-.-.-.-.-.-.-.-.-.-.-.-.-.-.-.-.-.-.-.-.-.-.-.-.-.-.-.-.-.-.-.-.-.-.-.
\subsection{Introduction}

The g711iplc directory contains an ANSI C implementation of the
G.711 Appendix I PLC algorithm. The C++ version of this algorithm is
in the g711iplc$\setminus$cpp\_cod directory. Sample test programs
read lost frame patterns in G.192 file format and apply the PLC
algorithm to audio files. The software in the g711iplc directory is
covered by a more restrictive copyright than the STL. See the
copyrght.txt file for details.

%Here is a summary of the files in the g77iplc directory:
%\begin{Descr}{70mm}
%    \item[\pbox{40mm}{lowcfe.h, lowcfe.c}] ANSI C implementation of the PLC algorithm
%    \item[\pbox{40mm}{g711iplc.c}] ANSI C test program for PLC algorithm
%    \item[\pbox{40mm}{plcferio.h, plcferio.c}] Routines for reading G.192 format pattern files
%    \item[\pbox{40mm}{error.h, error.c}] Error handling routine
%    \item[\pbox{40mm}{makefile.cl}]     Makefile file for Visual C++ on Windows
%    \item[\pbox{40mm}{copyrght.txt}] The software (c) Copyright
%\end{Descr}

%-.-.-.-.-.-.-.-.-.-.-.-.-.-.-.-.-.-.-.-.-.-.-.-.-.-.-.-.-.-.-.-.-.-.-.
\subsection{PLC Algorithm Implementation}

A detailed line by line description of the C++ code can be found in
section I.3 ``Algorithm description with annotated C++ code'' of G.711
Appendix I \cite{G711-appendix-I} and will not be repeated here. The
public interface functions that are called by applications are
covered. The C++ version is in the g711iplc$\setminus$cpp\_code
directory (files lowcfe.h and lowcfe.cc). The ANSI C version,
contained in the files lowcfe.h and lowcfe.c, is a translation of
the C++ code to C. The interface functions are the same for both
versions, with the exception that the C versions of the routines
take an extra argument for the data structure that is implicitly
passed to C++ member functions in the class instance data. As for
other STL modules, only the ANSI C version is compiled during
STL2005 building.

%\enlargethispage*{20mm}

%-.-.-.-.-.-.-.-.-.-.-.-.-.-.-.-.-.-.-.-.-.-.-.-.-.-.-.-.-.-.-.-.-.-.-.
\mysubsubsection{Constructor}

{\bf C++ syntax: }

{\tt \#include "lowcfe.h" \\
LowcFE  {\em{lc}}; // No argument constructor}

{\bf C syntax: }

{\tt \#include "lowcfe\_c.h" \\
g711plc\_construct({\em{LowcFE\_c*}}); /* explicit constructor call */}

{\bf Description: }

Before the PLC algorithm can be called the data structure containing
the algorithm's internal storage, such as the history buffer and
buffer pointers, must be initialized.

%-.-.-.-.-.-.-.-.-.-.-.-.-.-.-.-.-.-.-.-.-.-.-.-.-.-.-.-.-.-.-.-.-.-.-.
\mysubsubsection{Received Frames}

{\bf C++ syntax: }

{\tt void LowcFE::addtohistory(short {\em{*s}}); /* add a frame to
the history buffer */}

{\bf C syntax: }

{\tt void g711plc\_addtohistory({\em{LowcFE\_c*}}, short
{\em{*s}});}

{\bf Description: }

Frames of speech received from the transmitter are given to the PLC
algorithm with {\tt addtohistory} function. The argument {\em{s}}
points to a short array of length FRAMESZ (80 samples, or 10 ms)
that is used as both an input and output. Before the call is made
{\em{s}} is filled with the decoded G.711 data received from the
transmitter. On return, it contains the data that is output to the
listener. Addtohistory performs several operations. It stores the
input speech into the history buffer for use in generating the
synthetic signal if a loss occurs. If this is the first received
frame after a loss, an OLA is performed with the synthetic signal to
insure a smooth transition between the signals. In addition, it
delays the output so an OLA can be performed at the start of a loss.

%-.-.-.-.-.-.-.-.-.-.-.-.-.-.-.-.-.-.-.-.-.-.-.-.-.-.-.-.-.-.-.-.-.-.-.
\mysubsubsection{Lost Frames}

{\bf C++ syntax: }

{\tt void LowcFE::dofe(short {\em{*s)}};    /* synthesize speech
during loss */}

{\bf C syntax: }

{\tt void g711plc\_dofe({\em{LowcFE\_c*}}, short {\em{*s}});}

{\bf Description: }

If a frame is lost, the dofe routine is called. As with {\tt
addtohistory}, {\em{s}} is a pointer to short array of FRAMESZ
samples. With dofe, {\em{s}} is only an output. The PLC algorithm
fills {\em{s}} with the synthetic signal that conceals the missing
frame.

%-.-.-.-.-.-.-.-.-.-.-.-.-.-.-.-.-.-.-.-.-.-.-.-.-.-.-.-.-.-.-.-.-.-.-.
\mysubsubsection{Support Functions}

{\tt error}

{\bf Syntax: }

{\tt \#include "error.h" \\
void error\ttpbox{130mm}{(char {\em{*s}}, ...);}}

{\bf Description: }

Error handles fatal errors in the programs. The pattern string,
{\em{s}}, and optional following arguments should be in the format
of arguments accepted by the C library printf function. Error prints
its argument message on stderr and then exits the program. The error
function never returns.

{\tt readplcmask\_open}

{\bf Syntax: }

{\tt \#include "plcferio.h"\\
 void readplcmask\_open\ttpbox{130mm}{(readplcmask {\em{*r}},
char {\em{*fname}});}}

{\bf Description: }

The {\tt readplcmask\_open} function opens a G.192 format file
containing a packet loss pattern. {\em{fname}} is the file path. If
successfully opened, {\em{r}} contains the state information needed
for reading the patterns. {\tt readplcmask\_open} internally calls
the STL eid module to determine the type of the G.192 file and
select an appropriate reading function. If the open fails or an
unknown pattern is detected in the file, function {\tt error} is
called and {\tt readplcmask\_open} will not return.

{\tt readplcmask\_erased}

{\bf Syntax: }

{\tt \#include "plcferio.h"\\
int readplcmask\_erased\ttpbox{130mm}{(readplcmask {\em{*r}});}}

{\bf Description: }

{\tt readplcmask\_erased} reads the next value from the opened G.192
format pattern file. It returns 1 if the frame is lost and should be
concealed and 0 if the frame is ok. If the end of the G.192 file is
reached, the routine seeks back to the beginning of the file and the
pattern sequence is repeated. If an illegal value is found in the
G.192 file, the error function is called.

{\tt readplcmask\_close}

{\bf Syntax: }

{\tt \#include "plcferio.h"\\
void readplcmask\_close\ttpbox{130mm}{(readplcmask {\em{*r}})}}

{\bf Description: }

{\tt readplcmask\_close} is used to close a G.192 file that was
opened with {\tt readplcmask\_open}.

%-.-.-.-.-.-.-.-.-.-.-.-.-.-.-.-.-.-.-.-.-.-.-.-.-.-.-.-.-.-.-.-.-.-.-.
\subsection{Test Program}
%-...-...-...-...-...-...
\mysubsubsection{Test Program Usage}

The PLC algorithm is tested using g711iplc.c. The PLC test programs
take 3 file arguments:

\rulex{1cm}  {\tt g711iplc mask.g192 input.raw output.raw}

The {\tt mask.g192} file contains the lost frame pattern and should
be in the G.192 format as specified in the software tools library.
The g192, byte, and compact representations are supported. The G.192
file should contain only the frame headers words (G192\_SYNC or
G192\_FER, see softbit.h), and not the data words.

A frame corresponds to 10 ms, or 80 samples. If the lost frame
pattern file is shorter than the number of frames in the {\tt
input.raw} file, the program will roll-over back to the start of the
pattern file.  For example if the {\tt mask.g192} file contains the
binary data:
{\tt\small
\begin{verbatim}
    0x6B21 0x6B21 0x6B21 0x6B21 0x6B21,
    0x6B21 0x6B21 0x6B21 0x6B21 0x6B20
\end{verbatim}}
a 10\% uniform loss pattern will be applied to the whole file.
Erasures will occur at 90-100 ms, 190-200 ms, 290-300 ms ... in the
file.

While the algorithm is designed for packets containing 10ms of
speech, it can be applied to packetizations containing speech chunks
that are integer multiples of 10ms. For example, for a 10\% uniform
loss pattern with 20ms packetization one could use:

{\tt\small
\begin{verbatim}
    0x6B21 0x6B21 0x6B21 0x6B21 0x6B21,
    0x6B21 0x6B21 0x6B21 0x6B21 0x6B21,
    0x6B21 0x6B21 0x6B21 0x6B21 0x6B21,
    0x6B21 0x6B21 0x6B21 0x6B20 0x6B20
\end{verbatim}}
to cause erasures at 180-200ms, 380-400ms, 580-600ms, etc.

The input audio file, {\tt input.raw}, should contain header-less
16-bit binary data, sampled at 8 KHz, in the native byte order for
the machine running the test programs (big-endian on SPARC or MIPS,
little-endian on Intel). The test programs do not contain the G.711
encoder or decoder. If you have a G.711 bit-stream, it must be
decoded before the {\tt g711iplc} program is run.

The output audio file, {\tt output.raw}, also contains header-less
16-bit binary data. The PLC algorithm delays the output by 3.75 ms.
The test programs compensate for this delay by not outputting the
first 3.75 ms of the first packet. This way the input and output
files will be time aligned if they are overlaid in an audio waveform
editor. In addition, after the last full packet is input to the PLC
algorithm, an extra zero filled frame is input, and the first 3.75
ms of the corresponding output frame is sent to the output file. The
length of the output file will always be a multiple of the 10ms
frame size. If the input file length is not an integral number of
frames the last partial input frame will be discarded.

The test programs can also simulate a silence insertion algorithm
instead of the PLC algorithm with the {\tt -nolplc} option:

\rulex{1cm}  {\tt g711iplc -noplc mask.g192 input.raw output.raw}

Instead of calling the concealment algorithm the lost frames are
simply zero filled. This is helpful if you want to use a wave editor
to view the location of the missing frames.

Use the {\tt -stats} option to print out the number and percentage
of frames concealed in the processed file.

%-.-.-.-.-.-.-.-.-.-.-.-.-.-.-.-.-.-.-.-.-.-.-.-.-.-.-.-.-.-.-.-.-.-.-.
\mysubsubsection{Test Program Implementation}

A simplified version of the C++ test program is shown next. This
program does not support any options, such as -noplc, or compensate
for the algorithm delay, but demonstrates how the components work
together.

\newpage
{\tt\small
\begin{verbatim}
#include <stdlib.h>
#include <stdio.h>
#include <string.h>
#include "error.h"
#include "plcferio.h"
#include "lowcfe.h"

int main(int argc, char *argv[]) {
    FILE        *fi;        /* input file */
    FILE        *fo;        /* output file */
    LowcFE      fesim;      /* PLC simulation class */
    readplcmask mask;       /* error pattern file reader */
    short       s[FRAMESZ]; /* i/o buffer */

    argc--; argv++;
    if (argc != 3)
        error("Usage: g711iplc plcpattern speechin speechout");
    readplcmask_open(&mask, argv[0]);
    if ((fi = fopen(argv[1], "rb")) == NULL)
        error("Can't open input file: %s", argv[1]);
    if ((fo = fopen(argv[2], "wb")) == NULL)
        error("Can't open output file: %s", argv[2]);
    while (fread(s, sizeof(short), FRAMESZ, fi) == FRAMESZ) {
        if (readplcmask_erased(&mask))
            fesim.dofe(s);      /* lost frame */
        else
            fesim.addtohistory(s);  /* received frame */
        fwrite(s, sizeof(short), FRAMESZ, fo);
    }
    fclose(fo);
    fclose(fi);
    readplcmask_close(&mask);
    return 0;
}
\end{verbatim}
}

%-.-.-.-.-.-.-.-.-.-.-.-.-.-.-.-.-.-.-.-.-.-.-.-.-.-.-.-.-.-.-.-.-.-.-.
\subsection{Loss Pattern Conversion Utility}

The PLC directory includes a tool, {\tt asc2g192}, for converting
ASCII loss pattern files containing sequences of 0s and 1s into
G.192 format pattern files. In ASCII loss pattern files, a ``1''
represents a lost frame and a ``0'' represents a received frame. For
example, to create a 10\% uniform loss pattern with each loss being
10ms, use a text editor to create a text file called {\tt fe10.txt}:
{\tt\small
\begin{verbatim}
  0000000001
\end{verbatim}}
Then, convert it to the G.192 format for use by the g711iplc program
with the following command:
{\tt\small
\begin{verbatim}
  asc2g192 fe10.txt fe10.g192
\end{verbatim}}
Similarly, to create a 10\% uniform loss pattern with each loss
being 20ms (2 frames for each loss), create the text file {\tt
fe10\_2.txt} :
{\tt\small
\begin{verbatim}
  00000000000000000011
\end{verbatim}}
Then convert it to the G.192 format with:
{\tt\small
\begin{verbatim}
  asc2g192 fe10\_2.txt fe10\_2.g192
\end{verbatim}}
The {\tt asc2g192} conversion program ignores new lines and carriage
returns in the input file so the patterns can span multiple lines.
