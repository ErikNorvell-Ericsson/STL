%=============================================================================
\section{ITU-T STL G.728 Implementation}
%=============================================================================

This implementation of the G.728 algorithm is composed of source
files in several directories. The floating-point version of the
algorithm can be found in the directory {\tt g728/g728float}. The fixed-point
version is in {\tt g728/g728fixed}.
A third directory, {\tt g728/testvector}, is designed to hold the test
vectors for both versions of the algorithm.
The contents of the testvector directory are not part of the STL, but
can be obtained from G.728 Appendix I on the ITU-T website.
Similar data structures and interface functions are defined
for both the floating-point and fixed-point G.728 implementations.
The floating-point interface is discussed first.

\subsection {Floating-point G.728}

The source code for the floating-point version of G.728 resides in the
directory {\tt g728/g728float}. All public interface function
declarations and data structures needed to call the coder
can be found in the include
file {\tt g728.h}. The package includes a demonstration program, {\tt main.c},
that shows how to call the encoder, decoder, and decoder with
packet loss concealment. The demonstration program can
run the coder on the G.728 test vectors.

The floating-point version of the coder can be compiled
to use either double precision or single precision floating-point
arithmetic. If compiled with {\tt -DUSEDOUBLES}, g728.h defines the
{\tt Float} typedef as a C {\tt double}. If compiled with
with {\tt -DUSEFLOATS}, Float is a C {\tt float}. {\tt Float} is declared
in g728.h as:
{\tt\small
\begin{verbatim}
#ifdef USEDOUBLES
typedef double Float;
#endif
#ifdef USEFLOATS
typedef float  Float;
#endif
\end{verbatim}
}
Single precision runs faster. Double precision runs slower
but is more likely to give results that are bit-exact across
different machines. When compiling for use with the test vectors
double precision arithmetic should be used.

The include file, g728.h also defines a {\tt Short} type:

{\tt\small
\begin{verbatim}
typedef short  Short;
\end{verbatim}
}

to hold signed 16-bit integers.

\subsection {G.728 Floating-point Encoder}

g728.h defines a C structure, {\tt G728EncData}, to hold the encoder
state variables. Its contents are described in the G.728 standard.
Here we emphasize how to use it, rather than
its internal members. Before calling the encoder
{\tt G728EncData} should be initialized with a call to {\tt g728encinit}:

{\tt\small
\begin{verbatim}
void g728encinit(
     G728EncData *e    /* encoder state, initialize */
);
\end{verbatim}
}

Once initialized, frames of speech can be encoded with:

{\tt\small
\begin{verbatim}
void g728encode(
    Short       *index,  /* output indices */
    Float       *input,  /* input: speech */
    int	        sz,      /* input: size - must be multiple of IDIM */
    G728EncData *e       /* i/o: encoder state */	
);
\end{verbatim}
}

The {\tt input} speech should be an array of {\tt Float}s in the range of
-32768. to 32767. {\tt sz} is the dimension of the {\tt input} array,
and must be a multiple of the coder frame size, IDIM.
IDIM is defined to be 5 in g728.h.
{\tt index} is the output of the encoder. It is an array of code vectors
that are transmitted to the decoder.
{\tt index} has dimension sz / IDIM, as one
code vector will be output for each input frame of IDIM speech samples.
Only the least significant 10 bits of each index value are set.
The 3 gain index bits are in bits 0 to 2 (0 being the least
significant bit), and the 7 shape index bits are in bits 3 to 9.
Bits 10 to 15 in each index value are unused and contain zeros.
If an application desires to compactly transmit the
code vectors for multiple frames, it will have to extract the lower 10 bits
from each index word and pack the bits.

The floating-point G.728 software package includes utility functions for
converting between {\tt Short}s (16-bit integers) and {\tt Float}s:

{\tt\small
\begin{verbatim}
extern void g728_cpyr2i( /* convert Floats to Shorts,
                          * with saturation and rounding */
    Float *f,   /* input: float array */
    int   sz,   /* input: array size */
    Short *s    /* output: short array */
);
extern void g728_cpyi2r( /* convert Shorts to Floats */
    Short *s,   /* input: short array */
    int   sz,   /* input: array size */
    Float *f    /* output: float array
);
\end{verbatim}
}

These are used by the test program, main.c. A simple encoder loop
that reads in frames of 16-bit linear PCM, encodes them
with floating-point G.728, and outputs the index vectors is in
the following code fragment:

{\tt\small
\begin{verbatim}
#include "g728.h"

Short       ix;         /* output: index */
Short       s[IDIM];    /* input: 16-bit speech */
Float       f[IDIM];    /* converted to Float */
G728EncData e;          /* encode state */

g728encinit(&e);        /* initialize encoder state */
while (fread(s, sizeof(Short), IDIM, speechinf) == IDIM) {
        g728_cpyi2r(s, IDIM, f);        /* convert Short to Float */
        g728encode(&ix, f, IDIM, &e);   /* encode */
        fwrite(&ix, sizeof(Short), 1, indexf);  /* write to file */
}
\end{verbatim}
}

We have omitted the details of opening the files. The code in the
{\tt (mode == M\_ENC)} section of main.c is a more complicated example that
allows frame sizes that are multiples of IDIM, does more error checking,
and also allows byte swapping of the input and output files.

\subsection {G.728 Floating-point Decoder}

g728.h defines a C structure, {\tt G728DecData}, to hold the decoder
state variables. Many of the decoder state variables replicate
the state variables in the encoder. There are additional
sections for variables associated with the post filter and
packet loss concealment. As with the encoder, the {\tt G728DecData}
must be initialized before it can be used:

{\tt\small
\begin{verbatim}
void g728decinit(
    G728DecData *d    /* decoder state, initialize */
);
\end{verbatim}
}

Once initialized, frames of speech can be decoded with:

{\tt\small
\begin{verbatim}
void g728decode(
    Float       *speech, /* output: speech */
    Short       *index,  /* input: indices */
    int         sz,      /* input: size - must be multiple of IDIM */
    G728DecData *d       /* i/o: decoder state */	
);
\end{verbatim}
}

{\tt speech} points to the output speech array.
{\tt sz} is the dimension of the output speech array.
Like the encoder it must be
a multiple of the coder frame size, IDIM.
{\tt index} is a pointer to the array of input code vectors.
As with the encoder {\tt index} should be dimension sz / IDIM.
5 speech samples will be output for each input code vector.
The format of the index words is the same as in the encoder.

If the output speech is to be converted back to 16-bit linear PCM, the
conversion routine should take care of rounding and saturation.
The utility conversion routine, {\tt g728\_cpyr2i}, described above,
takes care of this. Here is a code fragment that can be used
to implement a floating-point decoder:

{\tt\small
\begin{verbatim}
#include "g728.h"

Short       ix;         /* input: index */
Float       f[IDIM];    /* output: speech, Float */
Short       s[IDIM];    /* output: speech, converted to 16-bit int */
G728DecData d;          /* decode state */

g728decinit(&d);        /* initialize decoder state */
while (fread(&ix, sizeof(Short), 1, indexf) == 1) {
    g728decode(f, &ix, IDIM, &d);    /* decode */
    g728_cpyr2i(f, IDIM, s);         /* convert to Short */
    fwrite(s, sizeof(Short), IDIM, speechoutf);
}
\end{verbatim}
}

Again we have omitted the details of opening the files.
The code in the {\tt (mode == M\_DEC)} section of main.c is a more
complicated example that allows frame sizes that are multiples of IDIM,
does more error checking, and also allows byte swapping of the input
and output arrays.

An additional interface routine, {\tt g728setpostf}, allows the
decoder's post-filter to be turned on or off. By default, the
post-filter is on after initialization, but it can be disabled
as required by several of the G.728 test vectors:

{\tt\small
\begin{verbatim}
void g728setpostf(
    int          i,     /* in: 1 postfilter on, 0 off */
    G728DecData *d      /* i/o: state variables */
);
\end{verbatim}
}

\subsection {G.728 Floating-point Encoder/Decoder}

The following program fragment shows how to run the encoder and
decoder together. It combines the fragment from the encoder, with the fragment
from the decoder, and uses the utility functions to convert between
Floats and Shorts. The loop sequence is: read in a frame of speech, convert
to Float, encode, decode, convert the output from Float to Short, and write
the resulting speech to a file:

{\tt\small
\begin{verbatim}
#include "g728.h"

Short       ix;         /* index */
Short       s[IDIM];    /* 16-bit speech vector*/
Float       f[IDIM];    /* Float speech vector */
G728EncData e;          /* encode state */
G728DecData d;          /* decode state */

g728encinit(&e);        /* initialize encoder state */
g728decinit(&d);        /* initialize decoder state */
while (fread(s, sizeof(Short), IDIM, speechinf) == IDIM) {
    g728_cpyi2r(s, IDIM, f);       /* convert Short to Float */
    g728encode(&ix, f, IDIM, &e);  /* encode */
    g728decode(f, &ix, IDIM, &d);  /* decode */
    g728_cpyr2i(f, IDIM, s);       /* convert to Short */
    fwrite(s, sizeof(Short), IDIM, speechoutf);
}
\end{verbatim}
}

The code that opens and closes the input and output files is omitted.

\subsection {G.728 Floating-point Decoder with Packet Loss Concealment}

The floating-point code supports the Packet Loss Concealment (PLC) algorithm
in G.728 Annex I. This algorithm, generates a synthetic
speech output at the decoder and maintains the decoder's internal
state variables if the input code vectors are lost in transmission.
This feature is implemented with two more functions. The first:

{\tt\small
\begin{verbatim}
void g728setfesize(
    int    plc25msec,   /* input: PLC frame size, in 2.5msec */
    G728DecData *d      /* i/o: decoder state */
);	
\end{verbatim}
}

sets the PLC frame size, in increments of 2.5msec (20 samples).
This function should be called after the G728DecData has been initialized
with a call to g728decinit. If the PLC frame size argument, plc25msec, is
set to 1, each lost frame will correspond to 20 samples. Plc25msec should
be in the range of 1 to 8, corresponding to a packet loss size of between
2.5 msec (20 samples) and 20 msec (160 samples). If the call to g728setfesize
is omitted, the decoder's default PLC frame size is 10 msec (80 samples).
The PLC algorithm uses state variables from the post-filter, so
the decoder's post-filter must not be deactivated when running the PLC algorithm.

The second PLC function:

{\tt\small
\begin{verbatim}
void g728decfe(
    Float       *speech,  /* output: speech */
    int         sz,       /* input: size, in samples, PLC framesize */
    G728DecData *d        /* i/o: decoder state */	
);
\end{verbatim}
}

is similar to the g728decode function except it takes no input index vector.
It serves two functions: it signals the decoder that the current frame
is lost, and it generates the synthetic output signal for the frame.
The sz argument is the number of samples in the PLC frame size.
Since there are 20 speech samples in 2.5 msec of speech, the sz argument to
g728decfe should be 20 times the plc25msec argument passed to g728setfesize.
For frames that do not have losses, the standard g728decode function should
be called.

A code fragment that shows how to call the PLC functions for a 10 msec PLC
frame size (80 samples) is shown below:

{\tt\small
\begin{verbatim}
#include "g728.h"

Short       ix[80/IDIM];/* input: index vectors */
Float       f[80];      /* output: Float array */
Short       s[80];      /* output: converted to 16-bit speech */
G728DecData d;          /* decode state */

g728decinit(&d);        /* initialize decoder state */
g728setfesize(4, &d);   /* 4 * 2.5msec frames = 10 msec */
while (fread(ix, sizeof(Short), 80/IDIM, indexf) == 80/IDIM) {
    if (ferasedin())
        g728decfe(f, 80, &d);           /* PLC decode */
    else
        g728decode(f, ix, 80, &d);     /* normal decode */
    g728_cpyr2i(f, 80, s);             /* convert to Short */
    fwrite(s, sizeof(Short), 80, speechoutf);
}
\end{verbatim}
}

As in the previous section, we have omitted the details of opening the files
as well as the function ferasedin() that returns a 1 if the current frame
should call the packet loss concealment code, and 0 if the standard decoding
routine should be called. The code in the {\tt (mode == M\_PLC)} section of
main.c shows a more complicated version of the PLC loop that supports
setting of the PLC frame size from the command line.

\subsection {Fixed-point G.728}

The source code for the fixed-point version of G.728 resides in the
directory {\tt g728/g728fixed}. All interface functions and data
structures needed to call the coder can be found in the
file {\tt g728fp.h}. The interface mirrors the floating-point
version of the codec, although the routines and data structures
have different names, and the input and output speech arrays are
{\tt Short}s instead of {\tt Float}s.
The fixed-point package also includes a demonstration program, {\tt main.c},
that shows how to call the encoder, decoder, and decoder with
packet loss concealment. The demonstration program can
run the coder on the test vectors. Please note that the fixed-point and
floating-point test vectors are different sets of files.

The fixed-point G.728 algorithm internally uses double precision floating-point
arithmetic to simulate some fixed-point operations. At the time this simulation
was written, this sped up the run-time and allowed the
coder to run in real-time on early Pentium-based computers. With today's
64 bit integer machines and faster processors, there are likely to be
better alternatives for simulating accumulator guard bits.
The fixed-point simulation code should generate bit-exact output for
all the test vectors.

Rather than repeating the interface presented in the floating-point section,
the following table lists the differences between the fixed-point and
floating-point versions.

\begin{center}
\begin{tabular} { | c | c | c | }
 \hline
Object & Floating-Point & Fixed-Point \\ \hline
Include File & g728.h & g728fp.h \\ \hline
Encoder State & G728EncData & G728FpEncData \\ \hline
Decoder State & G728DecData & G728FpDecData \\ \hline
Encoder State Initialize & g728encinit & g728fp\_encinit \\ \hline
Decoder State Initialize & g728decinit & g728fp\_decinit \\ \hline
Encoder & g728encode & g728fp\_encode \\ \hline
Decoder & g728decode & g728fp\_decode \\ \hline
Post-filter Disable & g728setpostf & g728fp\_setpostf \\ \hline
Packet Loss Size & g728setfesize & g728fp\_setfesize \\ \hline
Packet Loss Concealment & g728decfe & g728fp\_eraseframe \\ \hline
\end{tabular}
\end{center}

\subsection {G.728 Fixed-point Encoder/Decoder}

The program fragment below shows how to run the fixed-point encoder and
decoder together. Speech is input from a file, encoded, decoded, and
the output speech is written back to a file:

{\tt\small
\begin{verbatim}
#include "g728fp.h"

Short         ix;       /* index */
Short         s[IDIM];  /* 16-bit speech vector*/
G728FpEncData e;        /* encode state */
G728FpDecData d;        /* decode state */

g728fp_encinit(&e);     /* initialize encoder state */
g728fp_decinit(&d);     /* initialize decoder state */
while (fread(s, sizeof(Short), IDIM, speechinf) == IDIM) {
    g728fp_encode(&ix, s, IDIM, &e);   /* encode */
    g728fp_decode(s, &ix, IDIM, &d);   /* decode */
    fwrite(s, sizeof(Short), IDIM, speechoutf);
}
\end{verbatim}
}

This code closely resembles the code fragment for the floating-point code.
Note that the input and output speech arrays to the encoder and decoder
are Shorts so there is no need to call the Float to Short conversion routines.
Input and output speech arrays are 16-bit linear full range PCM.

\subsection {G.728 Demonstration Program}

Both the floating-point and fixed-point versions of the G.728 coder software
provide a demonstration program, main.c, that can be used to encode a
file, decode a file, decode with a file with packet losses, and
encode and decode a file in a single pass. Scripts are also
provided to use the demonstration program to run G.728 on the test vectors.
The test vectors themselves are not part of the STL.
The floating-point version compiles the program into a binary called {\tt g728}.
The fixed-point version compiles into a binary called {\tt g728fp}.
Since the programs provide an identical user
interface, we will only discuss {\tt g728}. If the examples below, substitute
{\tt g728fp} for {\tt g728} to use the fixed-point version of the coder.

The demonstration programs expect speech input and output files to be
header-less, and contain binary 16-bit linear full range PCM.
By default it is assumed that the files are in the native byte
order of the machine. The byte order of the
input and output files can be overridden on the command line with the
options {\tt -little} (for little-endian files) and {\tt -big}
(for big-endian files).
When the -little or -big options are given, the software first determines
if the current machine is a big-endian or little-endian machine.
If the machine has the same endian characteristics as the requested file,
no byte swapping is performed on the input and output files. If there is
a mismatch, byte swapping is performed. For example, the G.728 test vector
files are little endian files. Giving the -little option on Sun Sparc and
Mips processors (big-endian machines) will cause the input files to be byte
swapped before processing, and output data streams to be byte-swapped before
being written to files. On Intel processors, the -little option is a no-op.

Bit-streams are in the format of the ITU G.728 test vectors. For every 5 input
speech samples, a single 16-bit binary word is output. Only the least
significant 10 bits of each 16-bit output word are used.
The G.728 gain index (3 bits) resides in bits 0 to 2 (0 being the least
significant bit), and the shape index (7 bits) resides in bits 3 to 9.

Input PLC mask files are ASCII files that contain '0's and '1's.
A '1' implies the current frame is lost and should be concealed.
A '0' implies the normal decoding routine should be called for the current
frame. The PLC files may also contain newlines and return
characters that will be ignored. If any other character occurs in the file,
it is an error, and the program will exit with an error message.
If the PLC mask file reaches the end of the file
but there are still frames to decode, the PLC file will roll over and
seek back to the beginning of the file. For example, a PLC mask file containing
'00000000000000000001' can be used to introduce a 5% uniform loss pattern.

The demonstration program supports several other options. Options should
appear on the command line before the mode of the coder:

{\tt -nopostf}\\*
Turn off the decoder's post-filter. This is required to run many of
the test vectors. For normal operation the post-filter should be on.

{\tt -stats}\\*
PLC concealment option to print out statistics on how
many frames were processed and concealed.

{\tt -plcsize msec}\\*
set the PLC concealment frame size, in milliseconds. This is how much speech
will be concealed by a single lost packet. Only a limited set of values
are acceptable: 2.5, 5, 7.5, 10, 12.5, 15, 17.5 and 20. The default value
is 10 msec.

\subsection {G.728 Demonstration Program Modes}

The program has 4 modes: encode, decode, decode with plc, and encode and decode.
To run the encoder use:

{\tt\small
\begin{verbatim}
g728 enc speech.infile bitstream.outfile
\end{verbatim}
}

To run the decoder use:

{\tt\small
\begin{verbatim}
g728 dec bitstream.infile speech.outfile
\end{verbatim}
}

To run the combined encoder and decoder use:

{\tt\small
\begin{verbatim}
g728 encdec speech.infile bitstream.outfile speech.outfile
\end{verbatim}
}

To run the decoder with Packet Loss Concealment with the default 10msec
PLC frame size:

{\tt\small
\begin{verbatim}
g728 plc bitstream.infile plcmask.infile speech.outfile
\end{verbatim}
}

\subsection {G.728 Demonstration Program with Test Vectors}

Scripts ({\tt testall.sh} for Linux and Unix; {\tt testall.bat} for
Windows machines)
are provided for running the demonstration programs on the the G.728 test vectors. The scripts use binary compare functions to see if
the encoder and decoder output files differ
from the expected results.
For the fixed-point code an exact match is expected.
For the floating-point code an exact match is not guaranteed.
However, when compiled to use double-precision
arithmetic ({\tt -DUSEDOUBLES}) we have observed an exact match on
all platforms we have tested the code on.
If the code floating-point is compiled with the option {\tt -DUSEFLOATS},
differences in the test vectors should be expected.
For further details on the test vector verification
procedure when the outputs do not match, please refer to G.278
Appendix I.

A script ({\tt testplc.sh}, {\tt testplc.bat}) is also provided to run
the PLC algorithm on one of the test vectors. This program is only
to demonstrate how to correctly invoke the PLC algorithm, and does
not compare the output with file. It is not possible to compare the 
outputs to a fixed file since the PLC algorithm uses a
random number generator to extract the excitation when it determines
a lost packet frame is in an unvoiced region of speech.
Depending on where the packet losses occur, the output
may be different for each run on the same file.  
