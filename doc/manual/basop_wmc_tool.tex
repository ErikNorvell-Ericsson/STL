%----------------------------------------------------------------------
\section{Automatic complexity and memory instrumentation for floating-point C Code}
%----------------------------------------------------------------------

\subsection{Description}

The WMC tool allows for automatic complexity and memory instrumentation of speech and audio codecs written in the floating point format.
The tool inserts pseudo-instructions for counting the complexity of mathematical operations.
The weights of these operations are derived from the macros defined in Table \ref{tbl:flp-counters}.

The WMC tool also inserts instructions for measuring ROM and RAM consumption within the codec.
When running an instrumented codec the user receives a report detailing its complexity and maximum ROM and RAM consumption.

The WMC tool can only instrument {\tt .c} files respecting the ANSI/ISO 9899-1990 \cite{C90}.

There are some limitation as to the supported data types, expressions and logical operations that the WMC tool can successfully instrument.

\subsection{Supported instructions}

The following instructions that are recognized by the tool.

\subsubsection{Keywords}
{\tt
    if for while switch goto break continue return
}

\subsubsection{System Functions}
{\tt
    malloc calloc free
}

\subsubsection{Math Functions}
{\tt
    abs fabs labs

    floor

    sqrt sqrtf pow exp

    log log10

    cos sin tan acos asin atan atan2 cosh sinh tanh

    fmod

    min max Min Max MIN MAX

    sqr Sqr SQR square Square SQUARE

    sign Sign SIGN

    inv\_sqrt

    log\_base\_2 log2\_f

    round round\_f

    squant

    set\_min set\_max

    mac msu
}

\subsubsection{Recognized Data Types}
{\tt
    void bool char short int long float float **double Short Float Word16 Word16 **UWord16 Word32 Word32 **UWord32
    int8\_t int16\_t int16\_t **int32\_t int32\_t ** i nt64\_t uint16\_t uint32\_t uint64\_t
}

\subsection{Non instrumented instructions}

The following instructions are not instrumented by the tool.

\subsubsection{Keywords}
{\tt
    else
    do
    case
    default
    sizeof
}

\subsubsection{System functions}
{\tt
printf fprintf

fopen fclose fwrite fread

exit

assert

push\_indice set\_indice get\_indice exist\_indice

reset\_indices write\_indices read\_indices

read\_bitstream\_info reset\_stack push\_stack
}

\subsubsection{WMOPS library functions}
{\tt
    reset\_wmops push\_wmops pop\_wmops update\_wmops print\_wmops
}

\subsubsection{Counting Functions}
{\tt
    Dyn\_Mem\_Init Dyn\_Mem\_Exit Dyn\_Mem\_Exit\_noprint Dyn\_Mem\_In Dyn\_Mem\_Add Dyn\_Mem\_Out

    Sta\_Mem\_Init Sta\_Mem\_Exit Sta\_Mem\_Exit\_noprint Sta\_Mem\_Add

    P\_Dyn\_Mem\_Init P\_Dyn\_Mem\_Exit P\_Dyn\_Mem\_Exit\_noprint P\_Dyn\_Mem\_In P\_Dyn\_Mem\_Add P\_Dyn\_Mem\_Out

    P\_Sta\_Mem\_Init P\_Sta\_Mem\_Exit P\_Sta\_Mem\_Exit\_noprint P\_Sta\_Mem\_Add

    DYN\_MEM\_IN DYN\_MEM\_ADD DYN\_MEM\_OUT
}

\subsubsection{WMC\_Tool functions}
{\tt
    rsize

    Get\_Const\_Data\_Size

    Print\_Const\_Data\_Size

    print\_mem
}

\subsubsection{Ignored preprocessor Directives}
{\tt
    \#ifdef \#ifndef \#undef

    \#if \#else \#elif \#endif

    \#error \#pragma \#line \#warning
}

\subsection{Implementation}

WMC tool source code is composed of the following files:

\begin{itemize}
    \item {\tt wmc\_tool.cpp}: Main program
    \item {\tt c\_parser.cpp}: Routines for parsing C functions
    \item {\tt text\_utils.cpp}: Text and string processing functions
    \item {\tt output.cpp}: Warning and Error messages
\end{itemize}

The files {\tt wmc\_auto\_h.txt} and {\tt wmc\_auto\_c.txt} contain functions and definitions in the raw text format.
They are processed internally by the WMC tool and converted into {\tt wmc\_auto.h} and {\tt wmc\_auto.c}, respectively.
These files shall not be modified by an external program or converted manually.



\subsection{Usage}

\scriptsize
\begin{verbatim}

WMC tool v1.4 - STL2023

Mandatory arguments:
    space-separated list of filenames or directories with file mask, e.g. ./lib\_enc/array*.c
    note: if file mask is not specified *.c is assumed by default

Options:
    -h [--help]: print help
    -v [--verbose]: print warnings and other information messages
    -i [--info-only]: only print instrumentation information
    -d [--desinstrument]: desintrument only
    -m filename [--rom filename]: add statistics about ROM and RAM consumption
        note: filename shall point to a .c file containing the print\_mem() function
    -b [--no-backup]: no backup of original files
    -c dirname [--generate-wmc-files dirname]: copy wmc\_auto.h and wmc\_auto.c to a user-specified directory

\end{verbatim}
\normalsize

The WMC tool may be applied on individual {\tt .c} files or entire directories containing {\tt .c} files.
The examples below are provided for Unix-based systems.

To instrument a file named {\tt test.c} run the WMC tool as follows:

{\tt ./wmc\_tool test.c }

To instrument all {\tt .c} file in a directory named {\tt lib\_code} run the WMC tool as follows:

{\tt ./wmc\_tool "lib\_code/*.c" }

Note, that on Unix-based platforms it's necessary to use {\tt ""} quotes when entering file masks on the command-line.

The program ROM and table (const data) ROM is instrumented automatically in all {\tt .c} files.
The WMC tool inserts specific instructions and macros for tracking the maximum stack size and the maximum heap size.
To print out the statistics about the total ROM and RAM consumption within the codec the WMC tool shall be invoked with the {\tt -m} command-line option.
The statistics are then printed with the pre-defined {\tt print\_mem()} function.

The WMC tool generates two files {\tt wmc\_auto.h} and {\tt wmc\_auto.c} with the {\tt -c} command-line option.
These files shall be included in the instrumented codec to successfully interpret the instrumentation macros.
The best pratice is to include the file {\tt wmc\_auto.h} in all {\tt .c} files in the codec.


