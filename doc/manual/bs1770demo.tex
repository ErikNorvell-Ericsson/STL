
%=============================================================================
% ... THIS IS chapter{BS1770DEMO: The ITU-R BS.1770-4 Level measurement algorithm} ...
% Revisions
% Jan.2022 - First version
%=============================================================================
%=============================================================================
\chapter{BS1770DEMO: Example implementation of the ITU-R BS.1770-4 Level measurement algorithm}
%=============================================================================

\section{Introduction}

For a fair comparison of the conditions of a listening tests, it is essential
that the listening level is normalized. It is well-known that a higher listening
level often leads to an increased subjective score. The BS.1770-4 presents an algorithm
to measure the level of single and multichannel audio files, measured in dBLKFS. 
This is an implementation of this algorithm, along with the feature to apply scaling 
to obtain a desired dbLKFS level.

\section{Description of the Algorithm}

This example implementation follows the algorithm as described in \cite{BS1770}. The gating block
energy is first computed, where each gating block is 400 ms and extracted with 75\% overlap. This
means the advancement step is 100 ms. According to \cite{BS1770}, incomplete gating blocks should
be excluded from the computation. This means that the level measurement cannot be run for a signal
that is shorter than one gating block, i.e. shorter than 400 ms.

Before starting the energy calculation, the position dependent gain must be taken into account. 
According to \cite{BS1770}, loudspeaker that are positioned in the side region should have a gain
of $G=1.41$. The side region is defined by limits on azimuth $\theta$ and elevation $\phi$ according to:

  \[
    \left\{
       \begin{array}{ll}
         \phi < 30 ^{\circ} \\
         120 ^{\circ} \leq \theta \leq 60 ^{\circ} \\
       \end{array}
     \right.
  \]  

For loudspeakers outside of this range the gain is $G=1.0$, except for LFE channels which are excluded
from the loudness calculation by setting a gain of $G=0$. The loudspeaker configuration may be given on
the command line to \textbf{bs1770demo} as a configuration string with the same length as the number of
channels and where each character corresponds to a position category according to:

\begin{itemize}
    \item '0' - Not in the side region
    \item '1' - Within the side region
    \item 'L' - LFE channel
\end{itemize}

For example, a 5.1 configuration with [FL FR C LFE BL BR] may be configured with the string "000L11".
Two default configuration strings are given, depending on the number of channels. If the number of input
channels \texttt{nchan \textgreater 18}, the default configuration is "000L11000L11000000000000". If the 
number of input channels \texttt{nchan \textless = 18}, the configuration is set to "000L1100011000000". 
In both cases, the configuration is truncated to the number of channels. The configuration is parsed by
the function \texttt{parse\_conf} to form the vector of gain values \texttt{G[i]}.

Since the step between each gating block is 100 ms, the energy is first computed for each 100 ms
step and stored in the circular buffer \texttt{e\_tmp[4]} holding the last four values. For each
new gating block, the next 100 ms block is computed and then the energy of the gating block is found
by summing the energy of the four sub-blocks. For the beginning of the file, the first gating block
energy is computed only when the fourth sub-block is reached. This can be seen in the energy calculation
loop shown in the code below.

{\tt\small
\begin{verbatim}
    /* Obtain filtering and compute energy of gating blocks */
    for( n = 0, j = -3; j < n_gating_blocks; n++, j++ )
    {
        /* Read next sub-block */
        fread( input_short, sizeof( short ), STEP_SIZE * nchan, f_input );

        deinterleave_short2double( input_short, input, STEP_SIZE * nchan, nchan );

        /* Filter sub-block and store energy in circular buffer e_tmp */
        p_input = input;
        e_tmp[(n % 4)] = 0;
        for(i = 0; i < nchan; i++ )
        {
            iir2( p_input, p_input, STEP_SIZE, B1, A1, Bmem1 + 3 * i, Amem1 + 3 * i );
            iir2( p_input, p_input, STEP_SIZE, B2, A2, Bmem2 + 3 * i, Amem2 + 3 * i );
            e_tmp[(n % 4)] += G[i] * sumsq( p_input, STEP_SIZE );
            p_input += STEP_SIZE;
        }

        /* Compute energies of block j from 4 current sub-blocks in circular buffer, 
           excluding incomplete blocks */
        if( j >= 0 )
        {
            gating_block_energy[j] = 
                ( e_tmp[0] + e_tmp[1] + e_tmp[2] + e_tmp[3] ) / ((double)BLOCK_SIZE);
        }
    }
\end{verbatim}
}

The gating block energy is reused in the following loudness calculations. In case no output file is given on
the command line, the bs1770demo tool just outputs the measured loudness. If an output file is specified, a gain
factor is computed to obtain a target level. Since the applied gain factor may change the pattern of gating blocks
included in the loudness calculation, the gain factor is found using an iterative process implemented by the function
\texttt{find\_scaling\_factor}. The process halts when the ratio of the current and last gain factor is below a limit
\texttt{fabs( 1.0 - fac / last\_fac ) \textless RELATIVE\_DIFF}, where \texttt{RELATIVE\_DIFF = 0.0001} or when the maximum number of iterations \texttt{MAX\_ITERATIONS = 10}
has been reached.

{\tt\small
\begin{verbatim}
double find_scaling_factor(            /* o: scaling factor                 */
    const double *gating_block_energy, /* i: gating_block_energy            */
    const long n_gating_blocks,        /* i: Number of gating blocks        */
    const double lev,                  /* i: Target level                   */
          double *lev_input,           /* o: Input level                    */
          double *lev_obtained         /* o: Obtained level                 */
)
{
    long itr;
    double last_fac;
    double fac;
    double gated_loudness_final;

    last_fac = 100.0; /* Dummy init to trigger first iteration */
    fac = 1.0;
    itr = 0;
    while( (fabs( 1.0 - fac / last_fac ) > RELATIVE_DIFF) && (itr < MAX_ITERATIONS) )
    {
        /* Find scaling factor */
        gated_loudness_final = 
            gated_loudness_adaptive( gating_block_energy, fac, n_gating_blocks );
        last_fac = fac;
        fac *= pow( 10.0, (lev - gated_loudness_final) / 20.0 );
        if (itr == 0 )
        {
            *lev_input = gated_loudness_final;
        }
        itr++;
    }

    *lev_obtained = gated_loudness_final;
    return fac;
}
\end{verbatim}
}






\section{Usage of bs1770demo}

{\tt\small
\begin{verbatim}
bs1770demo [options] <input file> [<output file>]

<input file>      Input file,  16 bit PCM, 48 kHz
[<output file>]   Output file, 16 bit PCM, 48 kHz (Optional)

Options:
-nchan N          Number of channels [1..24] (Default: 1)
-lev L            Target level LKFS (Default: -26)
-conf xxxx        Configuration string:
                      '1' ldspk pos within |elev| < 30 deg, 60 deg <= |azim| <= 120 deg
                      'L' LFE channel (weight zero)
                      '0' otherwise
                      (Default conf nchan <= 18: 000L1100011000000)
                      (Default conf nchan  > 18: 000L11000L11000000000000)
\end{verbatim}
}
